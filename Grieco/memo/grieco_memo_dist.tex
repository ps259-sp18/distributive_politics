\title{Electoral impacts of Development Aid: Experimental Evidence from Sierra Leone}
\author{Kevin Grieco, UCLA}
\date{\today}
\documentclass[11pt]{article}
\usepackage[margin=0.5in]{geometry}
\begin{document}
\maketitle
\textbf{1. Overview} - 
This study evaluates the electoral effects of a recent public health development intervention in Sierra Leone on 2018 election result. In early 2017, 300 villages in Kono District --- Sierra Leone's most electorally competitive district--- were randomly selected to receive a Public Health intervention. The project was financed by international donors, implemented by the government of Sierra Leone, and supported by local authorities. Importantly, villages were randomly chosen to receive the intervention and politicians played no role in directing which villages received the intervention.\\

\textbf{2. Theory} - 
There is strong evidence that voters reward politicians for receiving resources that are distributed through clientelistic networks (Golden and Min 2013). However, it is not well understood how voters react to receiving randomly assigned development aid that politicians had no say in directing. On one hand, voters may regard development aid projects as part of programmatic government policy. Empirical evidence on electoral responses to programmatic policies is mixed (De la O, 2013; Imai et al 2017).  

On the other hand, while voters may regard development aid as independent of government action, beneficiaries may reward incumbents anyways (Hartmann 2017). Alternatively, independently delivered development aid may push beneficiaries towards opposition parties because aid frees voters from clientelist networks and allows voters to act on their true political preference (Blattman et al 2017).\\
This project also tests the capability of Paramount Chiefs to act as "vote brokers", where citizens vote for the Paramount Chief's preferred candidate when they expect the candidate to provide public goods though the Chief (Stokes et al 2013; Baldwin 2015; Koter, 2013).    \\

\textbf{3. Data Sources} \\ 
\underline{A. Exit Poll Survey} - The exit poll survey provides information on i) voting behavior for the four contested positions, ii) respondent’s perception of credit claiming behavior of politicians up for election, iii) social position of voter in community, iv) demographics and village of residence.\\
\underline{B. Paramount Chief Survey} - The Paramount Chief Survey elicits the electoral preferences of Paramount Chiefs by asking i) which party they feel is most capable of doing what is best for the people of Sierra Leone at a national level, ii) which Local Councilor/District Council Chairman/MP is most capable of brining development. \\

\textbf{4. Hypotheses}\\
\emph{Hypothesis 1:} Voters in villages that received the One Health intervention are more likely to vote for the incumbent candidate.\\

\emph{Hypothesis 2:} Voters are more likely to vote for incumbent politicians when they attribute the development project to incumbent government.\\

\emph{Hypothesis 3:} Citizens in highly politically connected villages are more likely to vote for the Paramount Chief's preferred candidate than citizens in non-politically connected villages.\\

\emph{Hypothesis 4:} Citizens in communities that have received a development project are more likely to vote with the Paramount Chief.

\end{document}

