\documentclass[]{article}
\usepackage{lmodern}
\usepackage{amssymb,amsmath}
\usepackage{ifxetex,ifluatex}
\usepackage{fixltx2e} % provides \textsubscript
\ifnum 0\ifxetex 1\fi\ifluatex 1\fi=0 % if pdftex
  \usepackage[T1]{fontenc}
  \usepackage[utf8]{inputenc}
\else % if luatex or xelatex
  \ifxetex
    \usepackage{mathspec}
  \else
    \usepackage{fontspec}
  \fi
  \defaultfontfeatures{Ligatures=TeX,Scale=MatchLowercase}
\fi
% use upquote if available, for straight quotes in verbatim environments
\IfFileExists{upquote.sty}{\usepackage{upquote}}{}
% use microtype if available
\IfFileExists{microtype.sty}{%
\usepackage{microtype}
\UseMicrotypeSet[protrusion]{basicmath} % disable protrusion for tt fonts
}{}
\usepackage[margin=1in]{geometry}
\usepackage{hyperref}
\hypersetup{unicode=true,
            pdftitle={Pre-Analysis Plan},
            pdfauthor={Jennifer Hamilton \& Ethan Christensen},
            pdfborder={0 0 0},
            breaklinks=true}
\urlstyle{same}  % don't use monospace font for urls
\usepackage{graphicx,grffile}
\makeatletter
\def\maxwidth{\ifdim\Gin@nat@width>\linewidth\linewidth\else\Gin@nat@width\fi}
\def\maxheight{\ifdim\Gin@nat@height>\textheight\textheight\else\Gin@nat@height\fi}
\makeatother
% Scale images if necessary, so that they will not overflow the page
% margins by default, and it is still possible to overwrite the defaults
% using explicit options in \includegraphics[width, height, ...]{}
\setkeys{Gin}{width=\maxwidth,height=\maxheight,keepaspectratio}
\IfFileExists{parskip.sty}{%
\usepackage{parskip}
}{% else
\setlength{\parindent}{0pt}
\setlength{\parskip}{6pt plus 2pt minus 1pt}
}
\setlength{\emergencystretch}{3em}  % prevent overfull lines
\providecommand{\tightlist}{%
  \setlength{\itemsep}{0pt}\setlength{\parskip}{0pt}}
\setcounter{secnumdepth}{0}
% Redefines (sub)paragraphs to behave more like sections
\ifx\paragraph\undefined\else
\let\oldparagraph\paragraph
\renewcommand{\paragraph}[1]{\oldparagraph{#1}\mbox{}}
\fi
\ifx\subparagraph\undefined\else
\let\oldsubparagraph\subparagraph
\renewcommand{\subparagraph}[1]{\oldsubparagraph{#1}\mbox{}}
\fi

%%% Use protect on footnotes to avoid problems with footnotes in titles
\let\rmarkdownfootnote\footnote%
\def\footnote{\protect\rmarkdownfootnote}

%%% Change title format to be more compact
\usepackage{titling}

% Create subtitle command for use in maketitle
\newcommand{\subtitle}[1]{
  \posttitle{
    \begin{center}\large#1\end{center}
    }
}

\setlength{\droptitle}{-2em}
  \title{Pre-Analysis Plan}
  \pretitle{\vspace{\droptitle}\centering\huge}
  \posttitle{\par}
\subtitle{Drivers of Distributive Preferences: Symbolic Politics vs Self-Interest
in Africa}
  \author{Jennifer Hamilton \& Ethan Christensen}
  \preauthor{\centering\large\emph}
  \postauthor{\par}
  \predate{\centering\large\emph}
  \postdate{\par}
  \date{May 11, 2018}


\begin{document}
\maketitle

\section{Introduction}\label{introduction}

In his 1957 classic \textit{An Economic Theory of Democracy}, Anthony
Downs adopts as a key assumption of his model that ``cititzens act
rationally in politics'' {[}@downs\_economic\_1957 36{]}. That is,
citizens cast votes aimed a maximizing their benefits from government.
Downs allows that utility can be derived from altruistic action, not
only from their material income or benefits. Nonetheless, research on
voting behavior continues to conceptualize voting as a rational and/or
self-intersted act. For instance, the oft-cited Meltzer-Richard model
posits that voters take into account their position in the income
distribution and vote to establish a system of taxation that maximizes
their private material self-interest {[}@meltzer\_rational\_1981{]}.

Mounting evidence suggests this idealized notion of a rational and
materially self-interested voter is not representative of reality.
Automatic emotional responses are now understood to play an important
role in information processing and decision making. For instance,
Charles S. Taber and Milton Lodge have collected evidence of ``hot
cognition,'' wherein political stimuli automatically and rapidly trigger
an affective charge from long-term memory; these affective charges
influence subsequent cognition {[}@lodge\_automaticity\_2005,
@morris\_activation\_2003{]}. Such affectively charged cognitions falls
short of the idealized rational cognition pervasive in political science
models.

Recognizing the role of affect in preference formation, David Sears and
coauthors developed a theory of symbolic politics
{[}@sears\_self-interest\_1980, @sears\_whites\_1979,
@lau\_self-interest\_1978{]}. Under this theory, citizens form
predispositions toward political objects or symbols early in life that
remain relatively stable over a lifetime. Later in life, exposure to
stimulus reminiscent of that political symbol triggers an affectively
charged response based on longstanding predispositions. The need for
cognitive consistency outweights present material self-interest. As a
result, ``Political attitudes\ldots{} are formed mainly in congruence
with long-standing values about society and the polity, rather than
short-term instrumentalities for satisfaction of one's current private
needs'' {[}@sears\_self-interest\_1980 671{]}. As a classic example,
individuals most involed in anti-Vietnam war protest did not have much
literal personal skin in game. They came from populations protected from
the draft -- young women, as well as young men who received higher
education exemptions from the draft. Symbolic predispositions better
explain their actions than material self-interest.

In order to gain empirical traction on the theory, Sears and coauthors
adopt a limited definition of self-interested attitudes as those that
are ``directed toward maximizing gains or minimizing losses to the
individual's tangible private well-being,'' excluding other forms of
utility such as value expression {[}@sears\_whites\_1979 369{]}. They
chose measure of self-interest that are as fine-grained and direct as
possible. Measuring self-interest with demographic variables, for
instance, is problematic because it fails to differentiate between
current self-interest and pre-adult socialization. Using U.S. polling
data, they found that symbolic predispositions (such as racial
prejudice, nationalism, ideology, or party identification) explained far
greater variation in attitudes toward the Vietnam war, unemployment
policy, national health insurance, busing, and law and order than did
material self-interest.

Our aim is to apply the symbolic politics theory to a new population. We
will do so by leveraging World Values Survey and Afrobarometer questions
that appopriately guage policy preference, symbolic predispositions, and
material self-interest on five topics: immigration, privatization,
education spending, healthcare spending, and ebola prevention spending.
We plan to investigate (1) whether symbolic predispositions and private
material self-interest each have any explanatory potential over policy
preferences, and (2) whether symbolic predispositions has greater
explanatory potential than material self-interest.

Owing to the difficult of manipulating either symbolic predispositions
or material self-interest, our investigation will not be causally
identified. Nonetheless, we believe this project could make several
important contributions to the literature:

\begin{itemize}
  \item To our knowledge, the theory of symbolic politics has not been applied outside the United States. Applying the theory in sub-Saharan Africa presents a new test for the theory in a new context. In particular, mayn psychological theories have not been adequately tested in non-WEIRD, i.e. non-Western, educated, industrialized, rich, democratic, societies [@henrich_weirdest_2010].
  \item Should symbolic politics prove resilient to this new context, the findings will allow better insight as to why ethnicity remains salient in African politics. Recent findings suggest that ethnic favoritism may not be as pervasive as once thought [@kasara_tax_2007, @kramon_who_2013, @burgess_value_2015]. Nonetheless, if ethnic group identity is a powerful symbolic attitude, political mobilization along ethnic lines may persist even where ethnic bias in service provision has faded. In addition, it may lead us to question models that assume rational voter behavior. For instance, the recent Metaketa I supposed that information on government performance would alter voter behavior. Expectations of such self-interested behavior may be misinformed if symbolic politics dominate voter preferences.
  \item Should symbolic politics not prove resilient in this new context, this finding will also be relevant to discussions over the salience of ethnic identity in African politics.
\end{itemize}

\section{Description of data}\label{description-of-data}

\section{Selection of the sample}\label{selection-of-the-sample}

\section{Hypotheses}\label{hypotheses}

\section{Variable construction}\label{variable-construction}

\section{Models and hypothesis
testing}\label{models-and-hypothesis-testing}

For each policy area, we will run an OLS regression with policy
preference as the dependent variable and with symbolic attitudes and
self-interest indices as the independent variables. We will also control
for demographic variables including gender, age, education, and
ethnicity. Thus, the model will take the form:

\[y_i = \beta_0 + \textbf{X}_{i1}\beta_1 + \textbf{X}_{i2}\beta_2 +  \textbf{X}_{i3}\beta_3 + \epsilon_i \]

where:

\begin{itemize}
  \item $\textbf{X}_{1i}$ represents a matrix of symbolic attitudes indicators and $\beta_1$ represents a vector of coefficients for each symbolic attitude indicator.
  \item $\textbf{X}_{2i}$ represents a matrix of self-interest indicators and $\beta_2$ represents a vector of coefficients for each self-interest indicator.
  \item $\textbf{X}_{i3}$ represents a matrix of control variables and $\beta_3$ represents a vector of coefficients for each control variable.
\end{itemize}

Note that symbolic attitudes and self-interest are thus \emph{not}
summary indices; rather, each indictor is entered individually into the
model. This replicates the original methodology used in Sears and
coauthors' series of symbolic politics papers {[}@sears\_whites\_1979,
@sears\_self-interest\_1980{]}.

For each model, we will conduct the following hypothesis tests:

\subsection{Joint F tests}\label{joint-f-tests}

To test whether symbolic attitudes and material self-interest each
contribute to the formation of policy preferences, we will use joint
F-tests. These test are NOT drawn from the Sears and coauthors papers,
which only consider the statistical significance of each covariate
individually.

\textbf{Symbolic Attitudes}

Recalling that \(\beta_1\) is a vector of coefficients on all symbolic
attitude covariates, the F-test for joint significance of symbolic
attitudes indicators has the following hypotheses:

\(H_0\): \(\beta_1 = \textbf{0}\)

\(H_A\): \(\beta_1 \neq \textbf{0}\)

Hence, the restricted model will be

\begin{equation}
y_i = \beta_0 + \textbf{X}_{i2}\beta_2 +  \textbf{X}_{i3}\beta_3 + \epsilon_i
\end{equation}

\textbf{Self-interest}

Recalling that \(\beta_2\) is a vector of coefficients on all
self-interest covariates, the F-test for joint significance of
self-interest indicators has the following hypotheses:

\(H_0\): \(\beta_2 = \textbf{0}\)

\(H_A\): \(\beta_2 \neq \textbf{0}\)

Hence, the restricted model will be

\[y_i = \beta_0 + \textbf{X}_{i1}\beta_1 +  \textbf{X}_{i3}\beta_3 + \epsilon_i\]

\subsection{R-squared comparison test}\label{r-squared-comparison-test}

In the original

\section{Multiple hypothesis testing}\label{multiple-hypothesis-testing}


\end{document}
