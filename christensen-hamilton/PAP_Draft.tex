\documentclass[]{article}
\usepackage{lmodern}
\usepackage{amssymb,amsmath}
\usepackage{ifxetex,ifluatex}
\usepackage{fixltx2e} % provides \textsubscript
\ifnum 0\ifxetex 1\fi\ifluatex 1\fi=0 % if pdftex
  \usepackage[T1]{fontenc}
  \usepackage[utf8]{inputenc}
\else % if luatex or xelatex
  \ifxetex
    \usepackage{mathspec}
  \else
    \usepackage{fontspec}
  \fi
  \defaultfontfeatures{Ligatures=TeX,Scale=MatchLowercase}
\fi
% use upquote if available, for straight quotes in verbatim environments
\IfFileExists{upquote.sty}{\usepackage{upquote}}{}
% use microtype if available
\IfFileExists{microtype.sty}{%
\usepackage{microtype}
\UseMicrotypeSet[protrusion]{basicmath} % disable protrusion for tt fonts
}{}
\usepackage[margin=1in]{geometry}
\usepackage{hyperref}
\hypersetup{unicode=true,
            pdftitle={Pre-Analysis Plan},
            pdfauthor={Jennifer Hamilton \& Ethan Christensen},
            pdfborder={0 0 0},
            breaklinks=true}
\urlstyle{same}  % don't use monospace font for urls
\usepackage{graphicx,grffile}
\makeatletter
\def\maxwidth{\ifdim\Gin@nat@width>\linewidth\linewidth\else\Gin@nat@width\fi}
\def\maxheight{\ifdim\Gin@nat@height>\textheight\textheight\else\Gin@nat@height\fi}
\makeatother
% Scale images if necessary, so that they will not overflow the page
% margins by default, and it is still possible to overwrite the defaults
% using explicit options in \includegraphics[width, height, ...]{}
\setkeys{Gin}{width=\maxwidth,height=\maxheight,keepaspectratio}
\IfFileExists{parskip.sty}{%
\usepackage{parskip}
}{% else
\setlength{\parindent}{0pt}
\setlength{\parskip}{6pt plus 2pt minus 1pt}
}
\setlength{\emergencystretch}{3em}  % prevent overfull lines
\providecommand{\tightlist}{%
  \setlength{\itemsep}{0pt}\setlength{\parskip}{0pt}}
\setcounter{secnumdepth}{0}
% Redefines (sub)paragraphs to behave more like sections
\ifx\paragraph\undefined\else
\let\oldparagraph\paragraph
\renewcommand{\paragraph}[1]{\oldparagraph{#1}\mbox{}}
\fi
\ifx\subparagraph\undefined\else
\let\oldsubparagraph\subparagraph
\renewcommand{\subparagraph}[1]{\oldsubparagraph{#1}\mbox{}}
\fi

%%% Use protect on footnotes to avoid problems with footnotes in titles
\let\rmarkdownfootnote\footnote%
\def\footnote{\protect\rmarkdownfootnote}

%%% Change title format to be more compact
\usepackage{titling}

% Create subtitle command for use in maketitle
\newcommand{\subtitle}[1]{
  \posttitle{
    \begin{center}\large#1\end{center}
    }
}

\setlength{\droptitle}{-2em}
  \title{Pre-Analysis Plan}
  \pretitle{\vspace{\droptitle}\centering\huge}
  \posttitle{\par}
\subtitle{Drivers of Distributive Preferences: Symbolic Politics vs Self-Interest
in Africa}
  \author{Jennifer Hamilton \& Ethan Christensen}
  \preauthor{\centering\large\emph}
  \postauthor{\par}
  \predate{\centering\large\emph}
  \postdate{\par}
  \date{May 11, 2018}


\begin{document}
\maketitle

\textit{Disclaimer}: The Afrobarometer website is technically
experiencing technical difficulties, which we expect to be resolved in
the near future. However, while we have been able to identify the text
of the questions we would like to use, we have not been able to identify
the standardized item number at present.

\section{Introduction}\label{introduction}

In his 1957 classic \textit{An Economic Theory of Democracy}, Anthony
Downs adopts as a key assumption of his model that ``cititzens act
rationally in politics'' {[}@downs\_economic\_1957 36{]}. That is,
citizens cast votes aimed a maximizing their benefits from government.
Downs allows that utility can be derived from altruistic action, not
only from their material income or benefits. Nonetheless, research on
voting behavior continues to conceptualize voting as a rational and/or
self-intersted act. For instance, the oft-cited Meltzer-Richard model
posits that voters take into account their position in the income
distribution and vote to establish a system of taxation that maximizes
their private material self-interest {[}@meltzer\_rational\_1981{]}.

Mounting evidence suggests this idealized notion of a rational and
materially self-interested voter is not representative of reality.
Automatic emotional responses are now understood to play an important
role in information processing and decision making. For instance,
Charles S. Taber and Milton Lodge have collected evidence of ``hot
cognition,'' wherein political stimuli automatically and rapidly trigger
an affective charge from long-term memory; these affective charges
influence subsequent cognition {[}@lodge\_automaticity\_2005,
@morris\_activation\_2003{]}. Such affectively charged cognitions falls
short of the idealized rational cognition pervasive in political science
models.

Recognizing the role of affect in preference formation, David Sears and
coauthors developed a theory of symbolic politics
{[}@sears\_self-interest\_1980, @sears\_whites\_1979,
@lau\_self-interest\_1978{]}. Under this theory, citizens form
predispositions toward political objects or symbols early in life that
remain relatively stable over a lifetime. Later in life, exposure to
stimulus reminiscent of that political symbol triggers an affectively
charged response based on longstanding predispositions. The need for
cognitive consistency outweights present material self-interest. As a
result, ``Political attitudes\ldots{} are formed mainly in congruence
with long-standing values about society and the polity, rather than
short-term instrumentalities for satisfaction of one's current private
needs'' {[}@sears\_self-interest\_1980 671{]}. As a classic example,
individuals most involed in anti-Vietnam war protest did not have much
literal personal skin in game. They came from populations protected from
the draft -- young women, as well as young men who received higher
education exemptions from the draft. Symbolic predispositions better
explain their actions than material self-interest.

In order to gain empirical traction on the theory, Sears and coauthors
adopt a limited definition of self-interested attitudes as those that
are ``directed toward maximizing gains or minimizing losses to the
individual's tangible private well-being,'' excluding other forms of
utility such as value expression {[}@sears\_whites\_1979 369{]}. They
chose measure of self-interest that are as fine-grained and direct as
possible. Measuring self-interest with demographic variables, for
instance, is problematic because it fails to differentiate between
current self-interest and pre-adult socialization. Using U.S. polling
data, they found that symbolic predispositions (such as racial
prejudice, nationalism, ideology, or party identification) explained far
greater variation in attitudes toward the Vietnam war, unemployment
policy, national health insurance, busing, and law and order than did
material self-interest.

Our aim is to apply the symbolic politics theory to a new population. We
will do so by leveraging World Values Survey and Afrobarometer questions
that appopriately guage policy preference, symbolic predispositions, and
material self-interest on five topics: immigration, privatization,
education spending, and ebola prevention spending. We plan to
investigate (1) whether symbolic predispositions and private material
self-interest each have any explanatory potential over policy
preferences, and (2) whether symbolic predispositions has greater
explanatory potential than material self-interest.

Owing to the difficult of manipulating either symbolic predispositions
or material self-interest, our investigation will not be causally
identified. Nonetheless, we believe this project could make several
important contributions to the literature:

\begin{itemize}
  \item To our knowledge, the theory of symbolic politics has not been applied outside the United States. Applying the theory in sub-Saharan Africa presents a new test for the theory in a new context. In particular, mayn psychological theories have not been adequately tested in non-WEIRD, i.e. non-Western, educated, industrialized, rich, democratic, societies [@henrich_weirdest_2010].
  \item Should symbolic politics prove resilient to this new context, the findings will allow better insight as to why ethnicity remains salient in African politics. Recent findings suggest that ethnic favoritism may not be as pervasive as once thought [@kasara_tax_2007, @kramon_who_2013, @burgess_value_2015]. Nonetheless, if ethnic group identity is a powerful symbolic attitude, political mobilization along ethnic lines may persist even where ethnic bias in service provision has faded. In addition, it may lead us to question models that assume rational voter behavior. For instance, the recent Metaketa I supposed that information on government performance would alter voter behavior. Expectations of such self-interested behavior may be misinformed if symbolic politics dominate voter preferences.
  \item Should symbolic politics not prove resilient in this new context, this finding will also be relevant to discussions over the salience of ethnic identity in African politics.
\end{itemize}

\section{Description of data}\label{description-of-data}

\section{Selection of the sample}\label{selection-of-the-sample}

\section{Hypotheses}\label{hypotheses}

\section{Variable construction}\label{variable-construction}

\section{Models and hypothesis
testing}\label{models-and-hypothesis-testing}

For each policy area, we will run an OLS regression with policy
preference as the dependent variable and with symbolic attitudes and
self-interest indices as the independent variables. We will also control
for demographic variables including gender, age, education, and
ethnicity. Thus, the model will take the form:

\begin{equation}
y_i = \beta_0 + \textbf{X}_{i1}\beta_1 + \textbf{X}_{i2}\beta_2 +  \textbf{X}_{i3}\beta_3 + \epsilon_i 
\end{equation}

where:

\begin{itemize}
  \item $\textbf{X}_{i1}$ represents a matrix of symbolic attitudes indicators and $\beta_1$ represents a vector of coefficients for each symbolic attitude indicator.
  \item $\textbf{X}_{i2}$ represents a matrix of self-interest indicators and $\beta_2$ represents a vector of coefficients for each self-interest indicator.
  \item $\textbf{X}_{i3}$ represents a matrix of control variables and $\beta_3$ represents a vector of coefficients for each control variable.
\end{itemize}

Note that symbolic attitudes and self-interest are thus \emph{not}
summary indices; rather, each indictor is entered individually into the
model. This replicates the original methodology used in Sears and
coauthors' series of symbolic politics papers {[}@sears\_whites\_1979,
@sears\_self-interest\_1980{]}.

For each model, we will conduct the following hypothesis tests:

\subsection{Joint F tests}\label{joint-f-tests}

To test whether symbolic attitudes and material self-interest each
contribute to the formation of policy preferences, we will use joint
F-tests. These test are NOT drawn from the Sears and coauthors papers,
which only consider the statistical significance of each covariate
individually.

\subsubsection{Symbolic Attitudes}\label{symbolic-attitudes}

Recalling that \(\beta_1\) is a vector of coefficients on all symbolic
attitude covariates, the F-test for joint significance of symbolic
attitudes indicators has the following hypotheses:

\(H_0\): \(\beta_1 = \textbf{0}\)

\(H_A\): \(\beta_1 \neq \textbf{0}\)

Hence, the restricted model will be

\begin{equation}
y_i = \beta_0 + \textbf{X}_{i2}\beta_2 +  \textbf{X}_{i3}\beta_3 + \epsilon_i
\end{equation}

\subsubsection{Self-interest}\label{self-interest}

Recalling that \(\beta_2\) is a vector of coefficients on all
self-interest covariates, the F-test for joint significance of
self-interest indicators has the following hypotheses:

\(H_0\): \(\beta_2 = \textbf{0}\)

\(H_A\): \(\beta_2 \neq \textbf{0}\)

Hence, the restricted model will be

\begin{equation}
y_i = \beta_0 + \textbf{X}_{i1}\beta_1 +  \textbf{X}_{i3}\beta_3 + \epsilon_i
\end{equation}

\subsubsection{Standard errors}\label{standard-errors}

For both joint F-test, we will use cluster-robust standard errors, with
primary sampling units as the cluster.

\subsection{R-squared comparison test}\label{r-squared-comparison-test}

In the original Sears and coauthors analysis compared the relative
contribution of symbolic attitudes and self-interest to policy
preferences by comparing the \(R^2\) contribution made by each set of
covariates. For instance, the contribution of symbolic attitudes will
be:

\[R^2_{\text{symbolic}} = R^2_1 - R^2_2\]

where

\begin{itemize}
  \item $R^2_{\text{symbolic}}$ represents the explanatory contribution of symbolic attitudes.
  \item $R^2_1$ represents the coefficent of determination for the full model, including symbolic attitude indicators, self-interest indicators, and control variables (equation 1).
  \item $R^2_2$ represents the coefficient of determination for the restricted model which excludes symbolic attitude indicators, but includes both self-interest indicators and control variables (equation 2).
\end{itemize}

Using the same technique, the contribution of self-interest will be:

\[R^2_{\text{self-interest}} = R^2_1 - R^2_3\]

where

\begin{itemize}
  \item $R^2_{\text{self-interest}}$ represents the explanatory contribution of self interest.
  \item $R^2_1$ represents the coefficent of determination for the full model, including symbolic attitude indicators, self-interest indicators, and control variables (equation 1).
  \item $R^2_2$ represents the coefficient of determination for the restricted model which excludes self-interest indicators, but includes both symbolic attitudes indicators and control variables (equation 2).
\end{itemize}

We will also use these estimates. However, we will also extend the
original analysis by estimating the standard errors of the \(R^2\)
contributions using bootstrapping, enabling us to assess more rigorously
whether contribution of either symbolic attitudes or self-interest
clearly outweights the other. Thus, we will test the following
hypotheses:

\(H_0\): \(R^2_{\text{symbolic}} - R^2_{\text{self-interest}} = 0\)

\(H_A\): \(R^2_{\text{symbolic}} - R^2_{\text{self-interest}} \neq 0\)

We will conduct 10,000 bootstraps to estimate standard errors. Our goal
is to maintain consistency between the original sampling procedure and
our bootstrap resampling method. However, we face certain practical
limitations.

\textit{Disclaimer: We are not yet settled on our bootstrapping procedure and will seek further advice before finalizing our PAP.}

\begin{itemize}
  \item Afrobarometer uses consistent sampling policies across countries and across survey rounds. For Afrobaromter questions, we will
  \subitem Divide the data back into country samples.
  \subitem Stratify the data by urban and rural PSUs within each country.
  \subitem Resample PSUs within each strata.
  \item The World Values Survey does not use consistent sampling methods across countries. As such, we resample according to the lowest common denominator among country sampling procedures:
  \subitem Divide the data back into country samples.
  \subitem Resample PSUs within each country.
\end{itemize}

\subsection{Sub-group analyses}\label{sub-group-analyses}

In their 1980 paper, Sears and coathors also examine results from a
series of sub-group analyses on the premise that self-interest may be
more prominent in preference formation under certain circumstances.
i.e.~within certain groups. They developed indices for political
sophistication, private-regarding values, perceived government
responsiveness, sense of political efficacy, and perceiving the issue as
a very important problem.

\textit{Looking for feedback: Should we subset within each country and then re-aggregate, or should we simply sub-set the merged data?}

\subsubsection{Political Sophistication}\label{political-sophistication}

This subgroup analysis is of greatest importance for Sears and
coauthors, as rationally self-interested behavior presumes the
availability of adequate information. In their 1980 analysis, they
measure political sophistication using a composite index of education,
basic political knowledge, self-reported interest in the ongoing
political campaign, and the number of media in which the campaign was
followed {[}@sears\_self-interest\_1980{]}. Following the discussion of
political sophistication measures by John Zaller
{[}@zaller\_nature\_1992{]}, we do not replicate Sears and coauthors'
composite index and instead use a summary index based solely on
political knowledge. These data are available for the Afrobarometer
question from Round 3 only (preference on school fees), so this is the
only policy on which we complete this sub-group analysis.

To measure political sophistication, we use an additive index based on
the following questions:

\begin{itemize}
  \item Can you tell me the name of: Your deputy name at the National Assembly?
  \item Can you tell me the name of: Your local government councilor?
  \item Can you tell me the name of: Your Deputy President?
  \item Do you happen to know which is the political party with the most seats at the National Assembly?
  \item Do you happen to know how many times a person can be elected president?
  \item Do you happen to know who is responsible of determining if a law is constitutional or not?
\end{itemize}

Please note that Afrobarometer questionnaires for individual countries
are customized to include appropriate terms for each country (e.g.~MP
versus deputy). We code 1 for the correct response, as identified by
Afrobarometer enumerators, and 0 for all other responses (including
``know but can't remember'').

Like Sears and coauthors {[}-@sears\_self-interest\_1980{]}, we will
split the participant pool at the median and rerun the tests separately
on each half (below median and above median).

\subsubsection{Private-regarding values}\label{private-regarding-values}

Children or Schwartz? WVS positive independent, hard work,
self-expression?; negative tolerance and respect for other people,
unselfishness, obedience

\subsubsection{Perceived government
responsiveness}\label{perceived-government-responsiveness}

Afrobarometer Round 3 contains several questions which allow us to
measure perceived government responsiveness, allowing us to subset our
data for policy preferences on school fees. We will create an additive
index based on the following questions, with responses to each question
normalized on a scale from 0 to 1.

\begin{itemize}
  \item How much of the time do think the following try their best to listen to what people like you have to say: Members of Parliament? Response options: (0) never, only sometimes, often, always (1)
  \item How much of the time do think the following try their best to listen to what people like you have to say: Local Government Councillors? Response options: (0) never, only sometimes, often, always (1)
  \item Taking the Problem you mentioned first [as the highest prioirity problem facing the country], how likely is it that the government will solve your most important problem within the next few years? Reponse options: (0) not at all likely, likely, not very likely, likely, very likely (1)
\end{itemize}

We will split the sample at the median, rerunning our analysis
separately on each half.

\textit{Again, should we split within each country or within the merged sample?}

\subsubsection{Sense of political
efficacy}\label{sense-of-political-efficacy}

Afrobarometer Round 3 contains a question on political efficacy,
allowing us to conduct this sub-group analysis for question. This
question reads: ``Do you agree or disagree with the following
statements? Politics and government sometimes seem so complicated that
you can't really understand what's going on.'' We divide respondents
into those who agree (includes both agree and strongly agree) and those
who disagree (both disagree and strongly disagree), dropping those who
respond with ``neither agree or disagree'' or ``don't know''. We
replicate our analysis on each sub-group.

\subsubsection{Issue importance}\label{issue-importance}

In both rounds of Afrobarometer that we use, respondents were asked to
identify up to the three most important problems facing the country that
the government should address. For our school fees question from
Afrobarometer Round 3, we will divide respondents into those who
identified education as an important problem (regardless of whether
first, second, or third choice) and those who did not, rerunning our
tests on each subset.

\% For our healthcare question from Afrobarometer Round 6, we will
divide respondents into those who identified health or sickness/disease
as an important problem (again, regardless or order stated) and those
who did not, rerunning our tests on each subset. Note that we do not
include those who identify AIDS as a priority in the former group, as we
found this response more ambiguous vis-a-vis the healthcare system:
addressing AIDS may involve better healthcare provisions (relevant to
the policy preference question), but alternatively may mean addressing
social stigma (not directly relevant to the policy preference question).

\%Round 6: priorities for investment? (Healthcare)

\section{Multiple hypothesis testing}\label{multiple-hypothesis-testing}

We will use the Benjamimi-Hochberg Procedure to address false discovery
through multiple hypothesis testing. In other words, we will:

\begin{itemize}
  \item Order the p-values for each of $m$ hypothesis tests such that $p_1 \leq p_2 \leq \ldots \leq p_m$.
  \item Maximize $k$ such that $p_i \leq \frac{\alpha \cdot k}{m}$. This value will be $k^*$.
  \item Reject all hypotheses $H_k$ for $k \leq k^*$
\end{itemize}


\end{document}
