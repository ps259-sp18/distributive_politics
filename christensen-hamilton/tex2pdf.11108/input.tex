\documentclass[]{article}
\usepackage{lmodern}
\usepackage{amssymb,amsmath}
\usepackage{ifxetex,ifluatex}
\usepackage{fixltx2e} % provides \textsubscript
\ifnum 0\ifxetex 1\fi\ifluatex 1\fi=0 % if pdftex
  \usepackage[T1]{fontenc}
  \usepackage[utf8]{inputenc}
\else % if luatex or xelatex
  \ifxetex
    \usepackage{mathspec}
  \else
    \usepackage{fontspec}
  \fi
  \defaultfontfeatures{Ligatures=TeX,Scale=MatchLowercase}
\fi
% use upquote if available, for straight quotes in verbatim environments
\IfFileExists{upquote.sty}{\usepackage{upquote}}{}
% use microtype if available
\IfFileExists{microtype.sty}{%
\usepackage{microtype}
\UseMicrotypeSet[protrusion]{basicmath} % disable protrusion for tt fonts
}{}
\usepackage[margin=1in]{geometry}
\usepackage{hyperref}
\hypersetup{unicode=true,
            pdftitle={Drivers of Distributive Preferences: Symbolic Politics vs Self-Interest in Africa},
            pdfauthor={Ethan Christensen \& Jenny Hamilton},
            pdfborder={0 0 0},
            breaklinks=true}
\urlstyle{same}  % don't use monospace font for urls
\usepackage{graphicx,grffile}
\makeatletter
\def\maxwidth{\ifdim\Gin@nat@width>\linewidth\linewidth\else\Gin@nat@width\fi}
\def\maxheight{\ifdim\Gin@nat@height>\textheight\textheight\else\Gin@nat@height\fi}
\makeatother
% Scale images if necessary, so that they will not overflow the page
% margins by default, and it is still possible to overwrite the defaults
% using explicit options in \includegraphics[width, height, ...]{}
\setkeys{Gin}{width=\maxwidth,height=\maxheight,keepaspectratio}
\IfFileExists{parskip.sty}{%
\usepackage{parskip}
}{% else
\setlength{\parindent}{0pt}
\setlength{\parskip}{6pt plus 2pt minus 1pt}
}
\setlength{\emergencystretch}{3em}  % prevent overfull lines
\providecommand{\tightlist}{%
  \setlength{\itemsep}{0pt}\setlength{\parskip}{0pt}}
\setcounter{secnumdepth}{0}
% Redefines (sub)paragraphs to behave more like sections
\ifx\paragraph\undefined\else
\let\oldparagraph\paragraph
\renewcommand{\paragraph}[1]{\oldparagraph{#1}\mbox{}}
\fi
\ifx\subparagraph\undefined\else
\let\oldsubparagraph\subparagraph
\renewcommand{\subparagraph}[1]{\oldsubparagraph{#1}\mbox{}}
\fi

%%% Use protect on footnotes to avoid problems with footnotes in titles
\let\rmarkdownfootnote\footnote%
\def\footnote{\protect\rmarkdownfootnote}

%%% Change title format to be more compact
\usepackage{titling}

% Create subtitle command for use in maketitle
\newcommand{\subtitle}[1]{
  \posttitle{
    \begin{center}\large#1\end{center}
    }
}

\setlength{\droptitle}{-2em}
  \title{Drivers of Distributive Preferences: Symbolic Politics vs Self-Interest
in Africa}
  \pretitle{\vspace{\droptitle}\centering\huge}
  \posttitle{\par}
  \author{Ethan Christensen \& Jenny Hamilton}
  \preauthor{\centering\large\emph}
  \postauthor{\par}
  \predate{\centering\large\emph}
  \postdate{\par}
  \date{April 23, 2018}


\begin{document}
\maketitle

\subsection{Theoretical Problem and
Contributions}\label{theoretical-problem-and-contributions}

Sub-optimal allocation of political goods may arise because citizens in
a democratic society express preferences for sub-optimal allocation.
Manin, Przeworski, and Stokes {[}-@przeworski\_democracy\_1999{]}
describe how such sub-optimal allocations can arise when poltiicians
pander to voter preferences that are not in the voters' best interests.
Within the American politics literature, the theory of symbolic politics
can explain why and when voters hold preferences contrary to their
objective material self-interests {[}@sears\_whites\_1979,
@sears\_self-interest\_1980{]}. Individuals acquire predispositions
toward political objects through socialization at a young age; thse
predispositions guide preferences later in life. In particular, symbolic
politics theory emphasizes racial prejudice, political ideology, and
partisan identification as drivers of policy preferences. As far as we
are aware, this theory has not yet been applied in an African context.
Especially in light of recent research that demonstrates the limits of
ethnic favoritism in the distribution of African political goods
{[}@kasara\_tax\_2007, @kramon\_who\_2013, @burgess\_value\_2015{]}, the
question of whether ethnic identity drives distributive preferences
rather than material self-interest merits investigation.

This project will make the following contributions to the literature:

\begin{itemize}
\item If symbolic politics explains greater variation in distributive preferences than material self-interest, one implication will be that welfare enhancement may not accompany democratization.
\item If material self-interest explains greater variation in distributive preference than symbolic predispositions like ethnic identity, we will clarify the debate over the salience of ethnic identity in African politics.
\item Our project will expand the literature on political psychology to developing countries, where it has not been extensively applied.
\end{itemize}

\subsection{Methods and Hypotheses}\label{methods-and-hypotheses}

We propose to use individual-level data collected in Afrobarometer's
sixth round survey (2014-2015) to study this topic. We will use two
questions to measure our dependent outcome, policy preferences. The
first question asked respondents in all 36 countries about willingness
to raise taxes if it results in better access to health care (Q65C). The
second question asked respondents in Liberia (Q86F-LIB) and Sierra Leone
(Q86G-SRL) about their willingness to allocate resources toward
combatting Ebola versus toward other prioirites. We will use other
Afrobarometer items to develop measure for material self-interest and
symbolic predisposition. For example, for the Ebola question, we will
gage self-interest using questions that ask whether respondents knew
infected individuals and whether Ebola disrupted their access to other
medical treatments. We will gage symbolic predispositions with questions
on strength of ethnic identity, party affiliation, and political
ideology. We will use a logit regression to differentiate the
explanatory potential of self-interest and symbolic predispositions,
just as Sears and coauthors did in their original analyses.

We expect that, overall, symbolic predispositions will explain greater
variation in African distributive preferences than material
self-interest. However, we expect the explanatory power of symbolic
predispositions to vary by country, especially as ethnic identities are
more salient in some countries than in others. In addition, we expect
that symbolic predispositions will explain greater variation in
preferences for the generalized healthcare question than the Ebola
question, as Ebola had concrete and highly observable impacts on
people's lives.


\end{document}
