\documentclass[]{article}
\usepackage{lmodern}
\usepackage{amssymb,amsmath}
\usepackage{ifxetex,ifluatex}
\usepackage{fixltx2e} % provides \textsubscript
\ifnum 0\ifxetex 1\fi\ifluatex 1\fi=0 % if pdftex
  \usepackage[T1]{fontenc}
  \usepackage[utf8]{inputenc}
\else % if luatex or xelatex
  \ifxetex
    \usepackage{mathspec}
  \else
    \usepackage{fontspec}
  \fi
  \defaultfontfeatures{Ligatures=TeX,Scale=MatchLowercase}
\fi
% use upquote if available, for straight quotes in verbatim environments
\IfFileExists{upquote.sty}{\usepackage{upquote}}{}
% use microtype if available
\IfFileExists{microtype.sty}{%
\usepackage{microtype}
\UseMicrotypeSet[protrusion]{basicmath} % disable protrusion for tt fonts
}{}
\usepackage[margin=1in]{geometry}
\usepackage{hyperref}
\hypersetup{unicode=true,
            pdftitle={Pre-Analysis Plan},
            pdfauthor={Ethan Christensen \& Jennifer Hamilton},
            pdfborder={0 0 0},
            breaklinks=true}
\urlstyle{same}  % don't use monospace font for urls
\usepackage{graphicx,grffile}
\makeatletter
\def\maxwidth{\ifdim\Gin@nat@width>\linewidth\linewidth\else\Gin@nat@width\fi}
\def\maxheight{\ifdim\Gin@nat@height>\textheight\textheight\else\Gin@nat@height\fi}
\makeatother
% Scale images if necessary, so that they will not overflow the page
% margins by default, and it is still possible to overwrite the defaults
% using explicit options in \includegraphics[width, height, ...]{}
\setkeys{Gin}{width=\maxwidth,height=\maxheight,keepaspectratio}
\IfFileExists{parskip.sty}{%
\usepackage{parskip}
}{% else
\setlength{\parindent}{0pt}
\setlength{\parskip}{6pt plus 2pt minus 1pt}
}
\setlength{\emergencystretch}{3em}  % prevent overfull lines
\providecommand{\tightlist}{%
  \setlength{\itemsep}{0pt}\setlength{\parskip}{0pt}}
\setcounter{secnumdepth}{0}
% Redefines (sub)paragraphs to behave more like sections
\ifx\paragraph\undefined\else
\let\oldparagraph\paragraph
\renewcommand{\paragraph}[1]{\oldparagraph{#1}\mbox{}}
\fi
\ifx\subparagraph\undefined\else
\let\oldsubparagraph\subparagraph
\renewcommand{\subparagraph}[1]{\oldsubparagraph{#1}\mbox{}}
\fi

%%% Use protect on footnotes to avoid problems with footnotes in titles
\let\rmarkdownfootnote\footnote%
\def\footnote{\protect\rmarkdownfootnote}

%%% Change title format to be more compact
\usepackage{titling}

% Create subtitle command for use in maketitle
\newcommand{\subtitle}[1]{
  \posttitle{
    \begin{center}\large#1\end{center}
    }
}

\setlength{\droptitle}{-2em}
  \title{Pre-Analysis Plan}
  \pretitle{\vspace{\droptitle}\centering\huge}
  \posttitle{\par}
\subtitle{Drivers of Distributive Preferences: Material Self-Interest
vs.~Expressive Benefit in Africa}
  \author{Ethan Christensen \& Jennifer Hamilton}
  \preauthor{\centering\large\emph}
  \postauthor{\par}
  \predate{\centering\large\emph}
  \postdate{\par}
  \date{May 11, 2018}

\usepackage{mathspec}
\usepackage{amsmath,amsthm}
\usepackage{dcolumn}
\usepackage{framed}
\usepackage[strings]{underscore}
\usepackage{array}

\begin{document}
\maketitle

\section{Introduction}\label{introduction}

Rewrite

\section{Description of data}\label{description-of-data}

In this study we plan to utilize data from the World Values Survey and
Afrobarometer surveys. The World Values Survey (WVS) is the largest and
longest-running non-commercial cross-national survey of beliefs and
values. The most recently completed wave surveyed participants from
almost 100 countries in 2011-2014. Each country-level survey has a
minimum of 1,200 respondents (ages 18-85), conducted through
face-to-face interviews. Depending on the country, sampling occurs
through either probability or a combination of probability and
stratified random sampling as noted in the documentation. From these
countries, we identify the five that are in sub-Saharan Africa: Ghana,
Nigeria, Rwanda, South Africa, and Zimbabwe.

Afrobarometer is a long running survey designed specifically to
understand political attitudes and behaviors in sub-Saharan Africa.
Respondents answer questions on a variety of topics including public
services, governance, identity, and political participation. There have
been six completed waves, with the most recent in 2014/2015 also
covering the most countries (36). Most country surveys have either 1,200
or 2,400 respondents. Each survey uses a clustered, stratified,
multi-stage, area probability sample. Occasionally a survey will
oversample a politically relevant sub-population in order to ensure a
large-enough subsample.

From these two sources we identify the four policy preference choices
available to us as listed above. In order for a policy preference
question to be included in our analysis, the relevant survey also had to
contain appropriate items to identify the self-interest and symbolic
attitudes of respondents, hence why so few policy preference items were
included.

For each topic we use different samples of countries based on the data
available. The relevant questions on immigration and privatization are
contained in the WVS and so is limited to those five countries. We limit
our analysis of education preferences to countries where there is not
compelling evidence of ethnic preference in education outcomes. If
ethnic preferences exist, then material self-interest is more difficult
to assess, as responses may rationally vary by ethnic group. We select
countries for inclusion using Franck and Rainer's results on systematic
bias in primary education attainment and literacy (2012). We limit our
analysis to only countries in which no more than one of their four
indicators attained statistical significance. This leaves us with five
countries: Benin, Malawi, Mali, Senegal, and Uganda.

\section{Hypotheses}\label{hypotheses}

Based on the original findings of the Sears and coauthors' papers, we
expect that both expressive benefits and and self-interest will be
jointly significant in the model for each question considered.

\section{Variable construction}\label{variable-construction}

\subsection{Immigration}\label{immigration}

\textit{Source}: World Values Survey Round 6

\textit{Countries included}: Ghana, Nigeria, Rwanda, South Africa, and
Zimbabwe

\textbf{Policy Preference}

(V46) When jobs are scarce, employers should give priority to people of
this country over immigrants.

\begin{itemize}

\item \textit{Original coding}: 1 = agree, 2 = neither, 3 = disagree, -5 = missing, -4 = not asked, -3 = not applicable, -2 = no answer, -1 = don't know 

\item \textit{Recode}: 1 = agree, 2 = neither, 3 = disagree, else NA

\end{itemize}

\textbf{Expressive Benefit}

\begin{enumerate}
  \item (V107) How much you trust: People of another nationality.
  \begin{itemize}
  \item \textit{Original coding}: 1 = trust completely, 2 = trust somewhat, 3 = do not trust very much, 4 = do not trust at all, -5 = missing, -3 = not applicable, -2 = no answer, -1 = don't know.
  \item \textit{Recode}: 1 = trust completely, 2 = trust somewhat, 3 = do not trust very much, 4 = do not trust at all, else NA.
  \item \textit{Expected sign}: negative.
  \end{itemize}
  \item (V16) Important child qualities: tolerance and respect for other people.
  \begin{itemize}
  \item \textit{Original coding}: 1= mentioned, 2 = not mentioned, -4 = not asked, -3 = not applicable, -2 = no answer, -1 = don't know.
  \item \textit{Recode}: 1= mentioned, 0 = not mentioned, else NA.
  \item \textit{Expected sign}: negative.
  \end{itemize}
\end{enumerate}

\textbf{Self-Interest}

\begin{enumerate}
  \item \textit{Personal Connection to Immigrant}: We will create a single indicator for personal connection to immigrants using the following items. If any of these items is answered as yes (1 = immigrant), personal connection to immigrant will be coded as one, else zero:
  \begin{itemize}
    \item (V243) Mother is an immigrant.  
    \item (V244) Father is an immigrant. 
    \item (V245) Respondent is an immigrant.  
  \end{itemize}
  We chose to fold these items into a single index in order to overcome potential multicollinearity. \\ \textit{Expected sign}: negative. 
  \item \textit{Employment status of primary household breadwinner} This indicator requires use of several WVS items: 
  \begin{itemize}
  \item (V229) Employment status \textit{Original coding}: 1 = full time, 2 = part time, 3 = self-employed, 4 = retired, 5 = housewife, 6 = students, 7 = unemployed, 8 = other, -4 = not asked, -3 = not applicable, -2 = no answer, -1 = don't know. 
  \item (V235) Are you the chief wage earner in your house? \textit{Original coding}: 1 = yes, 2 = no, -4 = not asked, -3 = not applicable, -2 = no answer, -1 = don't know.
  \item (V236) Is the chief wage earner employed now or not. \textit{Original coding}: 1 = yes, 2 = no, -4 = not asked, -3 = not applicable (Repondent is the chief wage earner), -2 = no answer, -1 = don't know.
  \end{itemize}
  We will code 1 if either the primary wage earner of the respondent's household (either respondent or otherwise) is employed in any capacity (full time, part time, or self-employed), else 0. \textit{Expected sign}: 
  \item \textit{Supplementary analysis}: We will include the following items in supplementary anaylsis. However, we exclude them from our prefered model because they are subjective measures of personal well-being, and therefore subject to endogeneity vis-a-vis value expression. 
  \begin{enumerate}
    \item (V59) Satisfaction with financial situation of household.
    \begin{itemize} 
    \item \textit{Original Coding}: scale from 1 = completely dissatisfied to 10 = completely satisfied, -4 = not asked in survey, -3 = not applicable, -2 = no answer or refused to answer, -1 = don't know.
    \item \textit{Recode}: scale from 1 to 10 as above, else NA
    \item \textit{Expected sign}: 
    \end{itemize}
    \item (V181) Worries: Losing my job or not finding a job.
    \begin{itemize}
    \item \textit{Original Coding}: 1 = very much, 2 = a great deal, 3 = not much,  4 = not much at all, -4 = not asked in survey, -3 = Inapplicable (neither has, nor seeks a job), -2 = no answer, -1 = don't know.
    \item \textit{Recode}: 1 = very much, 2 = a great deal, 3 = not much,  4 = not much at all, else NA.
    \item \textit{Expected sign}: positive.
    \end{itemize}
  \end{enumerate}
\end{enumerate}

\textbf{Controls}

\begin{enumerate}
  \item (V240) Sex, as coded by interviewer.
  \begin{itemize}
  \item \textit{Original coding}: 1= male, 2= female, -5 = missing/unknown, -4 = not asked in survey, -3 = not applicable, -2 = no answer, -1 = don't know. 
  \item \textit{Recode}: 1 = male, 0 = female, else NA.
  \end{itemize}
  \item (V241) Year of birth
  \begin{itemize}
  \item \textit{Original coding}: decade as indicated, -5 = missing/unknown, -4 = not asked in survey, -3 = not applicable, -2 = no answer, -1 = don't know.
  \item \textit{Recode}: 1 = 1900-1909, 2 = 1910-1919, 3 = 1920-1929, 4 = 1930-1939, 5 = 1940-1949, 6 = 1950-1959, 7 = 1960-1969, 8 = 1970-1979, 9 = 1980-1989, 10 = 1990-1999, 11 = 2000-2010, else NA.
  \end{itemize}
  \item (V248) What is the highest educational level that you have attained?
  \begin{itemize}
  \item \textit{Original coding}: 1= no formal education; 2= incomplete primary school; 3= complete primary school; 4= incomplete secondary school: technical/vocational type; 5= complete secondary school: technical/vocational type; 6= incomplete secondary school: university-preparatory type; 7= complete secondary school: university-preparatory type; 8= some university-level education, without degree; 9= university-level education, with degree; -5 = refused to answer; -4 = not asked; -3 = not applicable; -2 = no answer; -1 = don't know.
  \item \textit{Recode}: 1= no formal education; 2= incomplete primary school; 3= complete primary school; 4= incomplete secondary school: technical/vocational type; 5= complete secondary school: technical/vocational type; 6= incomplete secondary school: university-preparatory type; 7= complete secondary school: university-preparatory type; 8= some university-level education, without degree; 9= university-level education, with degree; else NA.
  \end{itemize}
  \item (V24) Most people can be trusted.
  \begin{itemize}
  \item \textit{Original Coding}: 1 = most people can be trusted, 2 = Need to be very careful, -4 = not asked in survey, -3 = not applicable, -2 = no answer, -1 = don't know.
  \item \textit{Recode}: 1 = most people can be trusted, 0 = need to be very careful, else NA.
  \end{itemize}
\end{enumerate}

\subsection{Privatization}\label{privatization}

\textit{Source}: World Values Survey Round 6

\textit{Countries included}: Ghana, Nigeria, Rwanda, South Africa, and
Zimbabwe

\textbf{Policy Preference}

(V97) Private vs.~state ownership of business

\begin{itemize}

\item \textit{Original coding}: scale from 1 = Private ownership of business and industry should be increased to 10 = Government ownership of business and industry should be increased, -5 = missing, -4 = not asked in survey, -3 = not applicable, -2 = no answer, -1 = don't know.

\item \textit{Recode}: scale from 1 = Private ownership of business and industry should be increased to 10 = Government ownership of business and industry should be increased, else NA.

\end{itemize}

\textbf{Expressive Benefit}

\begin{enumerate}
  \item (V96) Income inequality ideology. 
  \begin{itemize}
  \item \textit{Original coding}: scale from 1 = Incomes should be made more equal to 10 = We need larger income differences as incentives for individual effort, -5 = missing, -4 = not asked in survey, -3 = not applicable, -2 = no answer, -1 = don't know.
  \item \textit{Recode}: scale from 1 = Incomes should be made more equal to 10 = We need larger income differences as incentives for individual effort, else NA.
  \item \textit{Expected sign}: negative.
  \end{itemize}
  \item (V98) Government responsibility. 
  \begin{itemize}
  \item \textit{Original Coding}: scale from 1 = Government should take more responsiboity to ensure that everone is provided for to 10 = People should take more responsibility to provide for themselves, -5 = missing, -4 = not asked in survey, -3 = not applicable, -2 = no answer, -1 = don't know. 
  \item \textit{Recode}: scale from 1 = Government should take more responsiboity to ensure that everone is provided for to 10 = People should take more responsibility to provide for themselves, else NA.
  \item \textit{Expected sign}: negative.
  \end{itemize}
\end{enumerate}

\textbf{Self-interest}

\begin{enumerate}
 \item (V230) Sector of employment 
 \begin{itemize}
 \item \textit{Original coding}: 1 = government or public institution, 2 = private business or industry, 3 = private non-profit organization, 4 = autonomous/informal sector/other, -4 = not asked in survey, -3 = not applicable, -2 = no answer, -1 = don't know.
 \item \textit{Recode}: 1 = government or public institution; 0 = private business or industry, private non-profit organization, autonomous/informal sector/other; else NA.
 \item \textit{Expected sign}: positive. 
 \item \textit{Note}: this indicator is included on the basis of previous research on preferences over privatization [@battaglio_self-interest_2009].
 \end{itemize}
\end{enumerate}

\textbf{Controls}

\begin{enumerate}
  \item (V240) Sex, as coded by interviewer.
  \begin{itemize}
  \item \textit{Original coding}: 1= male, 2= female, -5 = missing/unknown, -4 = not asked in survey, -3 = not applicable, -2 = no answer, -1 = don't know. 
  \item \textit{Recode}: 1 = male, 0 = female, else NA.
  \end{itemize}
  \item (V241) Year of birth
  \begin{itemize}
  \item \textit{Original coding}: decades as indicated, -5 = missing/unknown, -4 = not asked in survey, -3 = not applicable, -2 = no answer, -1 = don't know.
  \item \textit{Recode}: 1 = 1900-1909, 2 = 1910-1919, 3 = 1920-1929, 4 = 1930-1939, 5 = 1940-1949, 6 = 1950-1959, 7 = 1960-1969, 8 = 1970-1979, 9 = 1980-1989, 10 = 1990-1999, 11 = 2000-2010, else NA.
  \end{itemize}
  \item (V248) What is the highest educational level that you have attained?
  \begin{itemize}
  \item \textit{Original coding}: 1= no formal education; 2= incomplete primary school; 3= complete primary school; 4= incomplete secondary school: technical/vocational type; 5= complete secondary school: technical/vocational type; 6= incomplete secondary school: university-preparatory type; 7= complete secondary school: university-preparatory type; 8= some university-level education, without degree; 9= university-level education, with degree; -5 = refused to answer; -4 = not asked; -3 = not applicable; -2 = no answer; -1 = don't know.
  \item \textit{Recode}: 1= no formal education; 2= incomplete primary school; 3= complete primary school; 4= incomplete secondary school: technical/vocational type; 5= complete secondary school: technical/vocational type; 6= incomplete secondary school: university-preparatory type; 7= complete secondary school: university-preparatory type; 8= some university-level education, without degree; 9= university-level education, with degree; else NA.
  \end{itemize}
  \item (V24) Most people can be trusted.
  \begin{itemize}
  \item \textit{Original Coding}: 1 = most people can be trusted, 2 = Need to be very careful, -4 = not asked in survey, -3 = not applicable, -2 = no answer, -1 = don't know.
  \item \textit{Recode}: 1 = most people can be trusted, 0 = need to be very careful, else NA.
  \end{itemize}
\end{enumerate}

\subsection{School fees}\label{school-fees}

\textit{Source}: Afrobarometer Round 3

\textit{Countries included}: Benin, Malawi, Mali, Senegal, Uganda

\textbf{Policy Preference}

(Q10) Which of the following statements is closest to your view?. Choose
Statement A or Statement B. A: It is better to have free schooling for
our children, even if the quality of education is low. B: It is better
to raise educational standards, even if we have to pay school fees.

\begin{itemize}

\item \textit{Original Coding}: 1=Agree Very Strongly with A, 2=Agree with A, 3=Agree with B, 4=Agree Very Strongly with B, 5=Agree with Neither, 9=Don’t Know, 98=Refused to Answer, -1=Missing Data

\item \textit{Recode}: 1=Agree Very Strongly with A, 2=Agree with A, 3=Agree with Neither, 4=Agree with B, 5=Agree Very Strongly with B, else NA.

\end{itemize}

\textbf{Expressive Benefit}

\begin{enumerate}
  \item (Q19) Let’s talk for a moment about the kind of society we would like to have in this country. Which of the following statements is closest to your view?. Choose Statement A or Statement B. A: People should look after themselves and be responsible for their own success in life. B: The government should bear the main responsibility for the well-being of people. 
  \begin{itemize}
  \item \textit{Original Coding}: 1=Agree Very Strongly with A, 2=Agree with A, 3=Agree with B, 4=Agree Very Strongly with B, 5=Agree with Neither, 9=Don’t Know, 98=Refused to Answer, -1=Missing Data. 
  \item \textit{Recode}: 1=Agree Very Strongly with A, 2=Agree with A, 3=Agree with Neither, 4=Agree with B, 5=Agree Very Strongly with B, else NA.
  \item \textit{Expected sign}: negative.
  \end{itemize}
\end{enumerate}

\textbf{Self-interest}

\begin{enumerate}
  \item (Q8F) Over the past year, how often, if ever, have you or your family gone without: School expenses for your children (like fees, uniforms or books)?
  \begin{itemize}
  \item \textit{Original Coding}: 0=Never, 1=Just once or twice, 2=Several times, 3=Many times, 4=Always, 7=No children, 9=Don’t Know, 998=Refused to Answer, -1=Missing Data
  \item \textit{Recode}:  
  \item \textit{Expected sign}: negative.
  \end{itemize}
  \item (Q73A) Have you encountered any of these problems with your local public schools during the past 12 months?. Services are too expensive / Unable to pay.
  \begin{itemize}
  \item \textit{Original Coding}: 0=Never, 1=Once or twice, 2=A few times, 3=Often, 7=No experience with public schools in the past twelve months, 9=Don’t Know, 98=Refused to Answer, -1=Missing Data.
  \item \textit{Recode}:  
  \item \textit{Expected sign}: negative.
  \end{itemize}
  \item \textit{School quality index} We will construct an additive index of school quality concerns, based on the following items:
  \begin{enumerate}
    \item (Q73B) Have you encountered any of these problems with your local public schools during the past 12 months?. Lack of textbooks or other supplies.
    \item (Q73C) Have you encountered any of these problems with your local public schools during the past 12 months?. Poor teaching.
    \item (Q73D) Have you encountered any of these problems with your local public schools during the past 12 months? Absent teachers.
    \item (Q73E) Have you encountered any of these problems with your local public schools during the past 12 months?. Overcrowded classrooms.
    \item (Q73F) Have you encountered any of these problems with your local public schools during the past 12 months? Poor conditions of facilities.
  \end{enumerate}
  Each of these questions has identical response options:
      \begin{itemize}
    \item \textit{Original coding}: 0=Never, 1=Once or twice, 2=A few times, 3=Often, 7=No experience with public schools in the past twelve months, 9=Don’t Know, 98=Refused to Answer, -1=Missing Data.
    \item \textit{Recode}:   
    \end{itemize}
  We expect a sdfl;asdfl
\end{enumerate}

\textbf{Controls}

\begin{enumerate}
  \item (Q90) What is the highest level of education you have completed?
  \begin{itemize}
  \item \textit{Original coding}: 0=No formal schooling, 1=Informal schooling (including Koranic schooling), 2=Some primary schooling, 3=Primary school completed, 4=Some secondary school/ High school, 5=Secondary school completed/High school, 6=Post-secondary qualifications, other than university e.g. a diploma or degree from a technical/polytechnic/college, 7=Some university, 8=University completed, 9=Post-graduate, 98=Refused to Answer, 99=Don’t Know, -1=Missing Data.
  \item \textit{Recode}: 0=No formal schooling, 1=Informal schooling (including Koranic schooling), 2=Some primary schooling, 3=Primary school completed, 4=Some secondary school/ High school, 5=Secondary school completed/High school, 6=Post-secondary qualifications, other than university e.g. a diploma or degree from a technical/polytechnic/college, 7=Some university, 8=University completed, 9=Post-graduate, else NA.
  \end{itemize}
  \item (Q101) Respondent's gender, as assessed by interviewer.
  \begin{itemize}
  \item \textit{Original coding}: 1=Male, 2=Female
  \item \textit{Recode}: 1 = male, 0 = female
  \end{itemize}
  \item (Q116B) Were the following services present in the primary sampling unit/enumeration area: School? 
  \begin{itemize}
  \item \textit{Original coding}: 0=No, 1=Yes, 9=Can’t determine, -1=Missing Data.\\ 
  \item \textit{Recode Scaling}: 0=No, 1=Yes, else NA.
  \end{itemize}
  \item (Q1) How old are you?
  \begin{itemize}
  \item \textit{Original coding}: 18-110 as indicated, 998=Refused to Answer, 999=Don’t Know, -1=Missing Data.
  \item \textit{Recode}: 18-110 as indicated, else NA
  \end{itemize}
  \item (Q79) What is your tribe? You know, your ethnic or cultural group. 
  \begin{itemize}
  \item \textit{Original coding}: : see AFB for full list, 990=National identity only, 995=Other, 997=Not Asked, 998=Refused, 999=Don’t know,-1=Missing Data.
  \item \textit{Recode}:
  \end{itemize}
\end{enumerate}

Also considering

\begin{enumerate}
  \item Think about the condition of [respondent’s identity group]. Do they have less, the same, or more influence in politics than other groups in this country? \textit{Scaling}: (1) much more, more, same, less, much less (0).  \textit{Expected sign}: positive.
  \item How often are [respondent’s identity group] treated unfairly by the government? \textit{Scaling}: (1) never, sometimes, often, always (0).  \textit{Expected sign}: positive.
  \item Let us suppose that you had to choose between being a [nationality] and being a [respondent’s identity group]. Which of the following statements best expresses your feelings? \textit{Scaling}: (1) ethnic ID only, ethnic ID more, national and ethnic, national ID more, national ID only (0).  \textit{Expected sign}: negative.
  \item How much do you trust each of the following types of people: [countrywomen] from other ethnic groups? \textit{Scaling}: (1) Not at all, just a little, somewhat, a lot (0).  \textit{Expected sign}: negative.
\end{enumerate}

\section{Model}\label{model}

\textit{Note: We are still deciding which model would be most appropriate, as response variables are not binary but they are bounded from 0 to 1. Logit might be better? We will determine this before finalizing our PAP.}

For each policy area, we will run an OLS regression with policy
preference as the dependent variable and with symbolic attitudes and
self-interest indices as the independent variables. We will also control
for demographic variables including gender, age, education, and
ethnicity. Thus, the model will take the form:

\begin{equation}
y_i = \beta_0 + \textbf{X}_{i1}\beta_1 + \textbf{X}_{i2}\beta_2 +  \textbf{X}_{i3}\beta_3 + \epsilon_i 
\end{equation}

where:

\begin{itemize}
  \item $\textbf{X}_{i1}$ represents a matrix of symbolic attitudes indicators and $\beta_1$ represents a vector of coefficients for each symbolic attitude indicator.
  \item $\textbf{X}_{i2}$ represents a matrix of self-interest indicators and $\beta_2$ represents a vector of coefficients for each self-interest indicator.
  \item $\textbf{X}_{i3}$ represents a matrix of control variables and $\beta_3$ represents a vector of coefficients for each control variable.
\end{itemize}

Note that symbolic attitudes and self-interest are thus \emph{not}
summary indices; rather, each indictor is entered individually into the
model. This replicates the original methodology used in Sears and
coauthors' series of symbolic politics papers (Sears, Hensler, and Speer
1979; Sears et al. 1980).

\section{Hypothesis tests}\label{hypothesis-tests}

For each model, we will conduct the following hypothesis tests:

\subsection{Joint F tests}\label{joint-f-tests}

To test whether symbolic attitudes and material self-interest each
contribute to the formation of policy preferences, we will use joint
F-tests. These test are NOT drawn from the Sears and coauthors papers,
which only consider the statistical significance of each covariate
individually.

\paragraph{Symbolic Attitudes}\label{symbolic-attitudes}

Recalling that \(\beta_1\) is a vector of coefficients on all symbolic
attitude covariates, the F-test for joint significance of symbolic
attitudes indicators has the following hypotheses:

\[\begin{aligned}
H_0 \  &: \beta_1 = \textbf{0}\\ H_A&: \beta_1 \neq \textbf{0}
\end{aligned}\]

Hence, the restricted model will be

\begin{equation}
y_i = \beta_0 + \textbf{X}_{i2}\beta_2 +  \textbf{X}_{i3}\beta_3 + \epsilon_i
\end{equation}

\paragraph{Self-interest}\label{self-interest}

Recalling that \(\beta_2\) is a vector of coefficients on all
self-interest covariates, the F-test for joint significance of
self-interest indicators has the following hypotheses:

\[\begin{aligned}
H_0\ &: \beta_2 = \textbf{0}\\ H_A &: \beta_2 \neq \textbf{0}
\end{aligned}\]

Hence, the restricted model will be

\begin{equation}
y_i = \beta_0 + \textbf{X}_{i1}\beta_1 +  \textbf{X}_{i3}\beta_3 + \epsilon_i
\end{equation}

\paragraph{Standard errors}\label{standard-errors}

For both joint F-test, we will use cluster-robust standard errors, with
primary sampling units as the cluster.

\subsection{R-squared comparison test}\label{r-squared-comparison-test}

In the original Sears and coauthors analysis compared the relative
contribution of symbolic attitudes and self-interest to policy
preferences by comparing the \(R^2\) contribution made by each set of
covariates. For instance, the contribution of symbolic attitudes will
be:

\[R^2_{\text{symbolic}} = R^2_1 - R^2_2\]

where

\begin{itemize}
  \item $R^2_{\text{symbolic}}$ represents the explanatory contribution of symbolic attitudes.
  \item $R^2_1$ represents the coefficent of determination for the full model, including symbolic attitude indicators, self-interest indicators, and control variables (equation 1).
  \item $R^2_2$ represents the coefficient of determination for the restricted model which excludes symbolic attitude indicators, but includes both self-interest indicators and control variables (equation 2).
\end{itemize}

Using the same technique, the contribution of self-interest will be:

\[R^2_{\text{self-interest}} = R^2_1 - R^2_3\]

where

\begin{itemize}
  \item $R^2_{\text{self-interest}}$ represents the explanatory contribution of self interest.
  \item $R^2_1$ represents the coefficent of determination for the full model, including symbolic attitude indicators, self-interest indicators, and control variables (equation 1).
  \item $R^2_2$ represents the coefficient of determination for the restricted model which excludes self-interest indicators, but includes both symbolic attitudes indicators and control variables (equation 2).
\end{itemize}

We will also use these estimates. Because the number of items for
self-interest and symbolic attitudes varies, we will use adjusted
\(R^2\) in all analysis; for instance, we would not want to bias our
results simply because the battery of symbolic attitude items is longer
than than the battery of self-interest items. In addition, we will
extend the original analysis by estimating the standard errors of the
\(R^2\) contributions using bootstrapping, enabling us to assess more
rigorously whether contribution of either symbolic attitudes or
self-interest clearly outweights the other. Thus, we will test the
following hypotheses:

\[\begin{aligned}
H_0 \ &: R^2_{\text{symbolic}} - R^2_{\text{self-interest}} = 0\\ H_A&: R^2_{\text{symbolic}} - R^2_{\text{self-interest}} \neq 0
\end{aligned}\]

We will conduct 10,000 bootstraps to estimate standard errors. Our goal
is to maintain consistency between the original sampling procedure and
our bootstrap resampling method. However, we face certain practical
limitations.

\textit{Disclaimer: We are not yet settled on our bootstrapping procedure and will seek further advice before finalizing our PAP.}

Afrobarometer uses consistent sampling policies across countries and
across survey rounds. For Afrobaromter questions, we will (1) divide the
data back into country samples, (2) stratify the data by urban and rural
PSUs within each country, and (3) resample PSUs within each strata. The
World Values Survey does not use consistent sampling methods across
countries. As such, we resample according to the lowest common
denominator among country sampling procedures (1) divide the data back
into country samples, and (2) resample PSUs within each country.

\section{Multiple hypothesis testing}\label{multiple-hypothesis-testing}

We will use the Benjamimi-Hochberg Procedure to address false discovery
through multiple hypothesis testing. In other words, we will:

\begin{itemize}
  \item Order the p-values for each of $m$ hypothesis tests such that $p_1 \leq p_2 \leq \ldots \leq p_m$.
  \item Maximize $k$ such that $p_i \leq \frac{\alpha \cdot k}{m}$. This value will be $k^*$.
  \item Reject all hypotheses $H_k$ for $k \leq k^*$
\end{itemize}

\section*{References}\label{references}
\addcontentsline{toc}{section}{References}

\hypertarget{refs}{}
\hypertarget{ref-franck_does_2012}{}
FRANCK, RAPHAËL, and ILIA RAINER. 2012. ``Does the Leader's Ethnicity
Matter? Ethnic Favoritism, Education, and Health in Sub-Saharan
Africa.'' \emph{The American Political Science Review} 106 (2):
294--325.

\hypertarget{ref-sears_whites_1979}{}
Sears, David O., Carl P. Hensler, and Leslie K. Speer. 1979. ``Whites'
Opposition to `Busing': Self-Interest or Symbolic Politics?'' \emph{The
American Political Science Review} 73 (2): 369--84.
doi:\href{https://doi.org/10.2307/1954885}{10.2307/1954885}.

\hypertarget{ref-sears_self-interest_1980}{}
Sears, David O., Richard R. Lau, Tom R. Tyler, and Harris M. Allen.
1980. ``Self-Interest Vs. Symbolic Politics in Policy Attitudes and
Presidential Voting.'' \emph{The American Political Science Review} 74
(3): 670--84.
doi:\href{https://doi.org/10.2307/1958149}{10.2307/1958149}.


\end{document}
