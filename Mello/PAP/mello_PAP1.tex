\documentclass[12pt,a4paper]{article}
\usepackage[latin1]{inputenc}
\usepackage{amsmath}
\usepackage{geometry}
\geometry{margin=1in}
\usepackage{amsfonts}
\usepackage{amssymb}
\usepackage{graphicx}
\usepackage{fancyhdr}
\usepackage{hyperref}
\usepackage{tikz}
\pagestyle{fancy}
\fancyhf{}
\rhead{Distributive Politics}
\lhead{Mello - Spring 2018}
\rfoot{Page \thepage}

\begin{document}


\title{Tying their own hands: why politicians vote on bills that increase control against themselves?}
\author{Fernando Mello, UCLA}
\date{\today}


\maketitle
\textbf{1. Research Question}\\

This project seeks to understand why politicians vote on bills that will eventually increase control against themselves, facilitating investigations and/or prosecution. In other words, why politicians tie their own hands. Mainly, I am interested in looking this question regarding Federal Deputies in Brazil, political actors known for their hability to distribute goods, bringing projects using budgetary amendments (Finan 2004, Bertholini et al 2017). Deputies reward municipalities based on political support. Particularly, I am interested in two questions:\\

\emph{Question 1:} Do voters punish politicians who vote against those laws?\\


\emph{Question 2:} Does the pressure of organized interest groups, mainly from the business community, influence the vote on those bills?\\


These are two different questions - and not necessarily mutually exclusive - that I believe are testable empirically. Bellow I present the motivation for the project and two designs to address these questions. Thus, I am not trying to adjudicate between these two hypotheses. In other words, I am not asking what causes Y, but if these Xs cause Y. \\
\\


\textbf{2. Motivation}\\


On April 5 2018, the Brazilian Supreme Court voted 6-5 to decide that the former President Luiz Inacio Lula da Silva (Lula) should to immediately start serving a 12-year sentence for corruption and money-laundering conviction for receiving a bribe from a construction company during his 2003-2010 presidency. In 2010, near the end of his second term, Lula enjoyed an approval rating of 87\% \footnote{President Barack Obama once called Lula the most popular politician of the world. For more: http://www.newsweek.com/brazils-former-president-loses-appeal-clouds-presidential-bid-790190}, making him one of the most popular politicians in the world. Most observers of Brazilian politics would agree that the investigation and eventual prosecution of Lula was only possible due only to a series of new laws approved during Lula`s own administration and that of his political heir and successor, Dilma Rousseff (2010-2016) (Nunes and Ranulfo Melo 2017).  

Lula is not the only politician who supported these laws only to run afoul of them. The same corruption scandal that felled Lula has also implicated more than two hundred current and former governors, mayors, federal deputies, and senators. These cases are based heavily on different types of evidence, but perhaps the most important has been testimony induced by plea bargains, a prosecutorial tool that was not allowed until 2013, when a new anticorruption law was passed (Lorezon 2017). 

The 2013 act was the latest in a series of important anti-corruption laws enacted since the end of the 1990s, the collective impact of which has been to increase transparency and weaken politicians` protections against corruption charges. 
Before that, in 2001, a constitutional change allowed federal politicians to be investigated. Since 1824, the Constitution granted formal immunity for deputies and senators. This immunity determined that deputies and senators could not be criminally tried without the permission of the respective House of Congress (Toffoli, 2016). But in 2001 the Constitutional Amendment No. 35, approved by Senators and deputies, altered this in such a way that it the permission was no longer necessary. According to Supreme Court Justices, with this constitutional reform, investigations have moved forward on a regular basis and criminal activities have been judged. Between 1988 \footnote{This was the year when the new Constitution was approved after the military dictatorship that ruled Brazil for three decades} and 2016, 628 criminal suits were processed at the Supreme Court, with 622 of them being initiated after passage of Constitutional Amendment No. 35/2001.

But corruption in Brazilian politics is not merely aimed at personal wealth - it is in, at least in part, a consequence of the high cost of running for (and keeping) public office, and of getting things done in a highly fragmented governmental structure (Juca, Melo and Renn 2016). The current scandal at Brazil's state-owned oil company, Petrobras, revolves around nearly US\$3 billion in bribes that contractors paid to Petrobras officials to secure construction and service contracts. Hundreds of millions of these dollars were diverted to the ruling Workers' Party and virtually all the parties in the country, which used the money to  finance its political campaigns (Fisman and Golden 2017). The case is known as Operation Carwash. If money is the key to political survival and effectiveness, why have politicians repeatedly chosen to tie their own hands, putting at risk their careers?  This is the puzzle of this project.\\


\textbf{3. Theory}\\


After conducting interviews with Brazilian politicians, bureaucrats, judges, and lawyers, I have come to believe variations in political support for anti-corruption laws might be a function of changes in the expected electoral costs of voting against those measures. Of course, politicians may vote for these laws because they do not fully understand their consequences or do not think that they are actually going to be harmed by them. However, for different observers of Brazilian politics this does not seem to be the case. It is possible that politicians voted this way because they were afraid of being punished by voters for appearing to oppose anti-corruption measures. Thus, they could have calculated that the downstream possibility of being prosecuted under the new law was not as important as the short-term need to win re-election. This is different from voters punishing politicians for corrupt acts (Ferraz and Finan 2013; Weitz-Shapiro and Winters 2016). According to this theory, politicians would prefer, in the short-term, to vote for these laws that are able to hurt them in the long term exactly because they seek the short-term reelection. 

The first theory can be summarized as the representation theory (Manin et al 1999). A common (and reliable) assumption posits that politicians have one primary goal: to get elected and reelected (Mayhew 1974). One problem with this literature is that, many times, it assumes the political system of the United States, with its single-seat electoral districts as the background for the interaction between politicians and potential supporters (Geddes, 1996). But different institutions create diverse outcomes. For instance, in Brazil, the electoral districts are the country's 27 federal units (26 states plus the federal district of Brasilia), and they elect between 8 and 70 representatives. A recent survey with 5003 respondents found out that 79\% of Brazilian voters do not remember for whom they voted for federal deputy in the last election, in 2014 (Moura 2018). If voters do not know who represents them in Congress, can they hold representatives accountable?

The second theory states that politicians actually vote on these laws when organized interest groups lobby for them. In 2013, Brazilians took over the street to protest against corruption. But the plea-bargaining law was a very exoteric topic. Groups of prosecutors and police officers organized to lobby for the bill. Organized business groups might also be a source of pressure. As shown by the Petrobras investigation, business groups use corruption as a tool to acquire contracts from the government. But they might also be better off without paying briberies. There is a collective action problem since individual companies might be able to take advantage by individually bribing politicians, instead of risking bearing the cost of supporting anti-corruption law. In certain situations, nevertheless, there might be an incentive to migrate from a high corruption equilibrium to a low corruption equilibrium (Fisman and Golden 2017). When the costs of individually paying bribes to politicians increase to a certain level, companies might prefer to comply with the rules or pressure for higher control. 

As stated by Fisman and Golden: "As we write this in the autumn of 2016, almost all members of Brazil's political elite have been exposed as deeply involved in large-scale corruption, as the world watches, astonished, by the daily revelations of the so-called Petrobras scandal. The country is in the midst of the largest corruption scandal ever to beset a democratic nation. After decades during which corruption was business as usual, there is the sudden threat of legal and political retribution."   \\


\textbf{4. Ideal experiments and observable implications}\\

The ideal experiment would be to randomly assign pressure of voters and interest groups before the voting of decisive bills such as the plea-bargaining law. Bellow, I present an idea to approximate that for the interest groups' perspective. However, when considering voters, I was not able to think about a way of doing this. Even so, I believe it is possible to test if voters report if they will punish politicians who voted against these laws.  

To summarize, there are two testable implications derived from the theories:\\


\emph{Hypothesis 1:} Voters will declare to vote less on politicians who voted against anti-corruption laws\\


\emph{Hypothesis 2:} Politicians will vote for transparency or anti-corruption laws when pressured by interest groups \\


\textbf{5. Empirical Strategy 1: Survey Experiment }\\
 
I will first address the strategy to test the representation hypothesis. Brazil has a general election this year, which will be held in October. A month before the election, I have the opportunity to include three questions in a national-representative survey with 2002 voters, conducted by Ibope (one of the two largest survey company in the country). Ibope has worked previously with other researchers such as Weitz-Shapiro and Winters. 

My proposal is to conduct a conjoint experiment, based on Hainmueller, Hopkins, and Yamamoto (2013). For instance, a recent survey shows that the main characteristics that Brazilian voters are looking for in a candidate for the Congress are honesty, transparency and not being involved in the Petrobras scandal (Carwash Operation). However, this survey does not ask about the anti-corruption laws that allowed the investigations or randomize any treatment. 


\begin{figure}[ht]
  \centering
  \scalebox{1}{\input{plot1.tex}}
  \caption{What do Brazilians want from Federal Deputies?} 
\end{figure}


The idea of the conjoint experiment is to try to isolate the effect of informing voters that the politician voted against the plea-bargaining law. My goal is to vary six characteristics, including party, experience, religion, capacity of delivering goods for the constituents, among others. The main treatment variables are informing the voters if the politician voted against or for the plea-bargaining law. I am also interested in confirming if voters punish politicians who had their names involved in the Petrobras scandal (Carwash Operation). I use three different options on purpose. A politician can only be cited in the media or during the investigations, without being formally charged. I want to check if this is sufficient to generate any punishment.

\begin{enumerate}
    	\item \textbf{Party:} PT, PSDB, PMDB, PP
        \item \textbf{Carwash Operation:} Was cited, was formally charged, is not cited 
        \item \textbf{Voted in favor of plea bargaining law:} Yes, No
        \item \textbf{Is responsible for projects of health and education to his city:} Yes, No
        \item \textbf{Education level:} no formal, 4th grade, 8th grade, high school, college degree, graduate degree
        \item \textbf{Religion:} Catholic, Evangelical, Jewish, or None  
    \end{enumerate}
    
    
These characteristics will be presented in the standard conjoint experiment way. According to the survey company, presenting the information in written messages is the best way to deliver information in Brazil, because even with less educated voters the enumerator can read the options. Also, according to Ibope, video or audio messages usually are not recommended.  

\begin{table}[]
\centering
\caption{\textbf{Please read the description of two potential candidates for Federal Deputies in the upcoming elections of October. Then, please indicate which of the two candidates would you personally prefer to see in Congress beginning in 2019.}}
\label{my-label}
\begin{tabular}{|l|l|l|}
\hline
                                                                & Candidate 1 & Candidate 2 \\ \hline
Party                                                           & PT          & PSDB        \\ \hline
Carwash Operation                                           & Name cited         &  Formally charged          \\ \hline
Voted in favor of plea bargaining law                           & No          & Yes         \\ \hline
Is responsible for projects of health and education to his city & Yes         & No          \\ \hline
Education level                                                 & College     & High School \\ \hline
Religion                                                        & Catholic    & Evangelic   \\ \hline

\end{tabular}
\end{table}


In this first question, voters are asked to choose between the two candidates, a "forced-choice" design that enables to evaluate the role of each attribute value in the assessment of one profile relative to another. After that, I will ask two extra questions, following the rating-based conjoint analysis: \textit{On a scale from 1 to 7, where 1 is you would absolute not vote for the candidate and 7 that you would definitely vote for the candidate, how would you rate candidate 1 (and candidate 2)? 
} 


I am worried about the sample size and the power of the experiment. Brazil has 146 million voters. When calculating the sample size necessary for a 4\% effect on a binary outcome - using the egap power calculator - I found that the sample size necessary is of 2065. 


\includegraphics[scale=0.4]{power1}

Nevertheless, Hainmueller, Hopkins, and Yamamoto have a sample size of 311 respondents for 186,624 possible profiles, when combining more characteristics, them the ones I am presenting. That is why I believe the sample size of 2002 respondents would be sufficient to estimate the average marginal component effect (AMCE). According to the authors, we can obtain the estimates of the AMCEs by running a single regression of the dependent variable on the combined sets of dummies for all candidate attributes.\\
  

\textbf{5. Empirical Strategy 2: Field experiment in the Brazilian Congress }\\

The second part of the project uses a field experiment to test the interest group's hypothesis. On April 17, 2018, the Brazilian Congress was expected to vote a new bill to regulate lobbying activity. This bill promises to increase transparency in the relationships between politicians and business representatives, and one of its declared goals is to increase the accountability of elected officials. For instance, it will be mandatory for deputies and senators to disclose any meetings with representatives from the private sector, expenditures related to those meetings, or presents and donations received. In fact, the first bill on regulating lobbying in the country was introduced in 1977. Since them different versions were presented by different congressmen, but the bill was never voted.

After several meetings, the NGO Movement Brazil Competitive - which assembles 51 of the largest companies in the country (such as IBM, Coca Cola, Ambev, Gerdau, among others) - has agreed to participate in the experiment. The president of the NGO supported the bill and was already lobbying politicians for that. After the meetings, we agreed that I would randomized the messages and send them. I blocked the treatment on different groups of parties. Assuming an effect of 0.11 percentage point, and using the EGAP power calculation, I would need 473 deputies of the Brazilian Congress to find an effect. Assuming an effect of 0.15 percentage point the sample necessary would be of 236 deputies out of 513. 

Brazil has almost 30 parties represented in the Congress. Blocking by parties would be virtually impossible. That is why I have blocked the deputies in three groups: government supporters, independents and opposition. The decision to block using groups of parties was made after analyzing a survey conducted with 185 deputies between February 19 and 20, 2018 \footnote{The survey was conducted by the Brazilian Political Scientist Leonardo Barreto, who gracefully shared the results}. In the survey, 75\% of the deputies said that they were in favor of the bill, but the results varied by groups of parties. Among the deputies that supported the government, 80\% said they were in favor of the bill, against 79\% of the independents and 69\% of the opposition.  At the same time, only 51\% of the deputies believed that the bill was going to be approved and transformed into law. The randomization was done within each block, with the distribution presented on table 2.

With IRB approval, two different messages were sent to the 513 federal deputies' cell phones. The control group received the following message:\\
 

\textit{Dear Deputy [insert his/her name], this is Claudio Gastal. The Chamber of Deputies will vote on the bill to regulate lobbying. As a citizen, I believe this is an essential measure to increase transparency in the relationship between politicians and private companies, generate higher accountability and provide more information for citizens. This is important to improve the relations between companies and politicians, especially after the recent events. I urge you to vote in favor of the bill. 
}\\

The treatment group received a similar message, with a slightly different beginning:\\
 

\textit{Dear Deputy [insert his/her name], this is Claudio Gastal. I am the President of the Movement Competitive Brazil, NGO that reunites 51 of the largest companies in the country. The Chamber of Deputies will vote on the bill to regulate lobbying. As a citizen, I believe this is an essential measure to increase transparency in the relationship between politicians and private companies, generate higher accountability and provide more information for citizens. This is important to improve the relations between companies and politicians, especially after the recent events. I urge you to vote in favor of the bill. Here a little bit more about our governance and performance.} \\

In the treatment group, 81.5\% of the deputies received the messages on their phones. In the control group, 78.8\% of the deputies received the message. 

\begin{table}[]
\centering
\caption{ Block Randomization}
\label{my-label}
\begin{tabular}{|l|l|l|l|}
\hline
            & Government & Opposition & Independent \\ \hline
Treated D=1 & 164        & 48         & 45          \\ \hline
Control D=0 & 165        & 47         & 44          \\ \hline
\end{tabular}
\end{table}


The problem is that the deputies never finished voting on the bill. Thus, I was not able to collect the outcome. After the discussion and the beginning of the section, the opposition was able to filibuster (they were protesting against the imprisonment of former president Lula) and the speaker of the house cancelled the voting. Now, there is no date for a new voting.

I would love to get feedback on the following questions and on any suggestions to improve the project:

\begin{enumerate}
    	\item Should I use the conjoint experiment or other design?
    		\item Am I correct when considering that I have power or am I making any mistake?
    		\item Should I include other characteristics such as sex, age or how many terms the deputy is in office?
    		\item Should I give up on the Congress experiment?
    		\item If I try to run it again, should I get new phone numbers for the deputies who did not receive the messages? 
    
    \end{enumerate}







\end{document}


