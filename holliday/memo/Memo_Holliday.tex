\documentclass[12pt,a4paper]{article}
\usepackage[latin1]{inputenc}
\usepackage{amsmath}
\usepackage{amsfonts}
\usepackage{amssymb}
\usepackage{graphicx}
\usepackage{fancyhdr}
\usepackage{hyperref}

\pagestyle{fancy}
\fancyhf{}
\lhead{Research Memo - PS 269 - Spring 2018}

\begin{document}
{\setlength{\parindent}{0cm}
\textbf{Name:} Derek Holliday\\

\textbf{Project Title:} The Redistributive Consequences of Electoral Competition: The Case of Federal Grant Expenditures in U.S. Municipalities\\

\textbf{Country/Setting:} U.S. Municipalities\\

\textbf{Data Sources:}
\begin{itemize}
	\item U.S. Department of Housing and Urban Development (HUD) - Community Development Block Grant Program (CDBG) - \href{https://www.hudexchange.info/programs/cdbg/cdbg-expenditure-reports/}{Expenditure Reports}
	\item Census/HUD Data for Low/Middle Income Percentages at Municipal Levels
\end{itemize}

\textbf{Data Units/Levels of Analysis:} U.S. Municipality-Years\\

\textbf{Theoretical Problem/Hypothesis:} Many theories of distributive politics suggest elected officials allocate public good expenditures based on electoral incentives; officials fearing sanction distribute funds in a manner that will win them votes. However, economists have also been perplexed by the relative ``stickiness'' of grant funding; grant funds tend to stay where such funding was targeted despite the economic inefficiencies of doing so (aka the ``flypaper effect''). This project will attempt to determine how the allocation of funds change with regard to proximity to elections,  with the main hypothesis that distributions ought to be distorted away from optimal social equity distributions as elections approach, but such distortion will be maximized when governments have the institutional power to unilaterally allocate funding.\\

\textbf{Importance:} A persistent problem within this vein of inquiry has been the endogenous determinants of revenue available for distribution. This project will take a step toward remedying the issue by utilizing Community Development Block Grants, which are annual entitlements to U.S. municipalities based on a general funding formula and provide trend data on the allocations of such funds.\\ 
}
\end{document}