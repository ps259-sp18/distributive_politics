\documentclass[12pt,a4paper]{article}
\usepackage[latin1]{inputenc}
\usepackage{amsmath}
\usepackage{geometry}
\geometry{margin=1in}
\usepackage{amsfonts}
\usepackage{amssymb}
\usepackage{graphicx}
\usepackage{fancyhdr}
\usepackage{hyperref}

\pagestyle{fancy}
\fancyhf{}
\lhead{Derek Holliday - Research Memo - PS 269 - Spring 2018}

\begin{document}
\begin{center}
	\textsc{The Redistributive Consequences of Electoral Competition: The Case of Federal Grant Expenditures in U.S. Municipalities}
\end{center}

\par Many theories of distributive politics suggest elected officials allocate public good expenditures based on electoral incentives; officials fearing sanction distribute funds in a manner that will win them votes. There is some debate on whether this distribution favors core or swing voters, but there is little disagreement on the existence of some distortion, However, economists have also noted the relative ``stickiness'' of grant funding; grant funds tend to stay where such funding was targeted despite the economic inefficiencies of doing so (aka the ``flypaper effect''). Given these two conflicting observable implications - the distortionary distribution of funds for electoral purposes versus the stickiness of funding for the projects they were meant for - what should be expected when elected officials are given relatively broad discretion over external funding designated for a certain end?

\par This project will attempt to consolidate economic and institutional theories of funding allocation by incorporating electoral and institutional pressures. Institutional pressures to allocate grant funding as intended are relatively constant through time; certain governments have more or less discretion over fund allocations by nature of their design or composition. Electoral pressures to distribute funds away from the intended recipients, however, may increase as election dates approach. Thus, I hypothesize that distributions ought to be distorted away from the intended distribution as elections approach, but such distortion will be maximized when governments have the institutional power to unilaterally allocate funding.

\par The context of this research will be U.S. municipalities receiving Community Development Block Grants (CDBGs) from the U.S. Department of Housing and Urban Development (HUD). CDBGs are awarded annually with allocations based on demographic formulas to municipal governments with the general intent of funding anti-poverty, housing, or other community development programs. As block grants, they have relatively little oversight or restrictions but do have a general intent, so they act as a semi-exogenous source of funds for municipalities to use. Data, then, will come from \href{https://www.hudexchange.info/programs/cdbg/cdbg-expenditure-reports/}{expenditure reports} listed publicly by HUD, which provide breakdowns of how the funds were used (for example, percent toward housing, infrastructure, or administrative costs) and to whom it was targeted (low or middle income). I will then be able to build estimates on ideal social equity distributions from census data on the income of the municipalities and conduct analysis at the municipality-year level, coding for election years.

\par This research is important for two main reasons. First, a persistent problem within distributive politics research has been the endogenous determinants of revenue available for distribution, and the context of this research is a step toward remedying such endogeneity concerns. Second, this research will help bridge the gap between distributive and managerial research veins. By focusing on both the economic and institutional underpinnings of popularly held theories in both veins, it may be possible to sooth extant discord between the two.

\end{document}