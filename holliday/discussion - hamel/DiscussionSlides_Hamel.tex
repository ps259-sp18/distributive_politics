\documentclass{beamer}
\mode<presentation> {
\usetheme{Madrid}
}

\usepackage{graphicx}

\title[Hamel (2018) Discussion]{Affluence and Influence? \\
A Discussion of Hamel (2018): Pre-Analysis Plan}

\author{Derek Holliday}
\institute[UCLA]{University of California - Los Angeles}
\date{\today}

\begin{document}

\begin{frame}
\titlepage
\end{frame}

\begin{frame}
\frametitle{Overview of Plan}
\begin{block}{Question}<1->
Do city governments provide fewer services to the poor (relative to the rich)?
\end{block}

\begin{block}{Data}<2->
311 Service Requests from Boston (2015 - Present), geocoded by census tract, coded for request fulfillment and fulfillment time.
\end{block}

\begin{block}{Empirical Strategy}<3->
OLS (fulfillment/time on census-tract median income) with neighborhood fixed effects.
\end{block}
\end{frame}

\begin{frame}
\frametitle{Theoretical Considerations}
\begin{itemize}
	\item Are the same people  responsible for policy input versus output? Are their motivations similar?
	\pause
	\item Are request fulfillments attributed to elected officials? Are avenues of recourse electoral?
	\pause
	\item Story may be less about contemporary distributions of services and more about historical bureaucratic holdovers.
\end{itemize}
\end{frame}

\begin{frame}
\frametitle{Data Considerations}
\begin{itemize}
	\item Have to believe 311 calls are a common avenue through which heterogeneous income levels access similar public services. Is this true?
	\pause
	\begin{itemize}
		\item Potential solution: code calls be topic of request.
		\item Nature of the request may also be indicative of past service failures.
	\end{itemize}
	\pause
	\item Data on who ends up fulfilling requests?
\end{itemize}
\end{frame}

\begin{frame}
\frametitle{Empirical Strategy Considerations}
\begin{itemize}
	\item Understandable tension between precise identification and capturing lingering variance between tracts
	\pause
	\begin{itemize}
		\item Fixed effects won't capture anything that varies between census tracts within neighborhoods
		\pause
		\item ... but it's unclear whether you want to tie down everything that isn't income.
	\end{itemize}
	\pause
	\item Consider non-OLS model types (Logit for binary, Poisson/Negative Binomial for time data).
\end{itemize}
\end{frame}

\begin{frame}
\frametitle{Conclusion}
Two ways this could go: policy evaluation or underpinnings of service inequities. Empirical strategy will depend on which avenue you choose.
\end{frame}

\end{document}