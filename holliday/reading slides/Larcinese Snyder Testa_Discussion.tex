\documentclass{beamer}
\mode<presentation> {
\usetheme{Madrid}
}

\usepackage{graphicx}

\title[Larcinese, Snyder, Jr., and Testa (2012)]{Testing Models of Distributive Politics using Exit Polls to Measure Voters' Preferences and Partisanship. Larcinese, Snyder, Jr., and Testa (2012)}

\author{Derek Holliday}
\institute[UCLA]{University of California - Los Angeles}
\date{\today}

\begin{document}

\begin{frame}
\titlepage
\end{frame}

\begin{frame}
\frametitle{Motivation/Question}
Models of distributive politics give use three hypotheses:
\begin{enumerate}
	\item Swing Voters
	\item Battleground States
	\item Core Voters
\end{enumerate}
Do any of these hypotheses find support in the United States?
\end{frame}

\begin{frame}
\frametitle{Empirical Problem}
Using vote shares (even lagged vote shares) to predict distribution of federal funds is at least partially endogenous. 
\begin{enumerate}
	\item Lagged votes are a function of past promises, correlated with present distributions
	\item Budget allocations are sluggish
	\item Omitted variables correlated with votes and budget decisions
\end{enumerate}
\pause
Simulation demonstrates significantly biased results from using vote shares.
\pause
Solution: use exit polling data for ideological distribution within states.
\end{frame}

\begin{frame}
\frametitle{Empirical Strategy}
Authors regress distribution of federal funds on measures for proportion of independents (swing), closeness of previous vote share (battleground), and proportion of copartisans (core) with demographic controls and state fixed effects.
\end{frame}

\begin{frame}
\frametitle{Results}
No luck. Neither the swing voters no battlegrounds hypothesis show results robust to different model specifications or with the expected sign. \medskip

\pause
Modest support for core voters, but the size of the relationship is small (1 percent increase in partisan support correlated with increase of \$4.30 in per capita spending).
\end{frame}

\begin{frame}
\frametitle{Discussion}
Finding may be a function of the institutional organization of fund distribution in the United States (especially with regard to the President) or a general practice in the United States to use campaigning rather than distributive goods to swing voters.
\end{frame}

\end{document}